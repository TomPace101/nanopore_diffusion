%not a stand-alone document; intended for inclusion in a larger document

\subsection{Unhomogenized Fickian Diffusion Equation}\label{subsec:unhom_fick}

%-------------------------------------------------------------------------------
\subsubsection{Governing Differential Equation}\label{subsubsec:unhom_fick_gov}

\textcolor{red}{\textbf{TODO}}: need equation reference


For particle flux defined as 

\begin{equation}
\vec{j} = - D_{\mathrm{bulk}} \vec{\nabla} c
\end{equation}

The Fickian diffusion equation can be written:
\begin{equation}
\frac{\partial c}{\partial t} = - \vec{\nabla} \cdot \vec{j}
\end{equation}

where:

$\begin{array}{rcl}
c & = & c(x,y,z) = \text{the particle concentration field (number of particles per unit volume)} \\
t & = & \text{time} \\
\vec{j} & = & \vec{j}(x,y,z) = \text{the particle flux field (number of particles per unit area per unit time)} \\
D & = & \text{the implicit diffusion constant of the solvent (unit area per unit time)}
\end{array}$

For constant $D_{\mathrm{bulk}}$, this can be written as

\begin{equation}
\frac{\partial c}{\partial t} = D_{\mathrm{bulk}} \nabla^2 c
\end{equation}

We seek the steady-state solution, defined by
$\frac{\partial c}{\partial t} = 0$ at all points in the problem domain.
Thus, we seek to solve

\begin{equation}
D_{\mathrm{bulk}} \nabla^2 c = 0
\end{equation}

subject to boundary conditions.

Assuming $D_{\mathrm{bulk}}$ is any nonzero constant, this is clearly the same as

\begin{equation}
\nabla^2 c = 0
\end{equation}


%-------------------------------------------------------------------------------
\subsubsection{Weak Form}\label{subsubsec:unhom_fick_weak}

For solution in \texttt{FEniCS}, a weak form of this equation is required.
Multiplying by a test function $v$ and integrating over the problem domain,

\begin{equation}
\int_{\Omega} \left(\nabla^2 c \right) v \,\mathrm{d}^3x = 0
\end{equation}

Using the product rule
\begin{equation}
\vec{\nabla} \cdot \left( v \vec{\nabla} c \right) =
v \left(\vec{\nabla} \cdot \vec{\nabla} c \right) + \vec{\nabla}c \cdot \vec{\nabla}v =
v \left(\nabla^2 c \right) + \vec{\nabla}c \cdot \vec{\nabla}v
\end{equation}

\begin{equation}
\left(\nabla^2 c \right) v =
\vec{\nabla} \cdot \left( v \vec{\nabla} c \right) - \vec{\nabla}c \cdot \vec{\nabla}v
\end{equation}

to integrate by parts, the equation becomes
\begin{equation}
\int_{\Omega} \left(\nabla^2 c \right) v \,\mathrm{d}^3x =
\int_{\Omega} \vec{\nabla} \cdot \left( v \vec{\nabla} c \right) \,\mathrm{d}^3x
- \int_{\Omega} \left( \vec{\nabla}c \cdot \vec{\nabla}v \right) \,\mathrm{d}^3x =0
\end{equation}

Applying the divergence theorem (Gauss's theorem):
\begin{equation}
\int_{\partial\Omega} \left( \hat{n} \cdot \vec{\nabla} c \right) v\,\mathrm{d}s
- \int_{\Omega} \left( \vec{\nabla}c \cdot \vec{\nabla}v \right) \,\mathrm{d}^3x = 0
\end{equation}

\begin{equation}
\int_{\Omega} \left( \vec{\nabla}c \cdot \vec{\nabla}v \right) \,\mathrm{d}^3x =
\int_{\partial\Omega} \left( \hat{n} \cdot \vec{\nabla} c \right) v\,\mathrm{d}s
\end{equation}

In this case, we have both Dirichlet and von Neumann boundary conditions:
there are some boundary surfaces where the concentration is known,
and others where the particle flux is known.
Specifically, known concentrations are applied at both the top and bottom of the model,
and the particle flux must be zero for all other boundary surfaces.
Defining $\Gamma_D$ as the Dirichlet boundary surfaces,
and $\Gamma_N$ as the von Neumann boundary surfaces, we obtain

\begin{equation}
\int_{\Omega} \left( \vec{\nabla}c \cdot \vec{\nabla}v \right) \,\mathrm{d}^3x =
\int_{\Gamma_D} \left( \hat{n} \cdot \vec{\nabla} c \right) v\,\mathrm{d}s
+\int_{\Gamma_N} \left( \hat{n} \cdot \vec{\nabla} c \right) v\,\mathrm{d}s
\end{equation}

For the Dirichlet boundary surfaces (where the concentration is known),
the test function $v$ must be equal to zero,
as the variation of the unknown function must be zero at points where the function is known.

Furthermore, a flux of zero requires that the derivative of the concentration in a direction
normal to the boundary surface is zero.
That is, $\hat{n} \cdot \vec{\nabla} c = 0$ at all points on the von Neumann boundaries.

Therefore, the governing equation is simply
\begin{equation}
\int_{\Omega} \left( \vec{\nabla}c \cdot \vec{\nabla}v \right) \,\mathrm{d}^3x = 0
\end{equation}

In terms of \texttt{FEniCS}, this means that the bilinear form is
$a(c,v)=\left( \vec{\nabla}c \cdot \vec{\nabla}v \right) \,\mathrm{d}^3x$,
and the linear form is constant, zero.


%-------------------------------------------------------------------------------
\subsubsection{Expected Results}\label{subsubsec:unhom_fick_expected}

The effective diffusion constant of Section \ref{subsec:D_eff}
was derived by simplification of the governing equation
to an unhomogenized Fickian diffusion equation.
Consequently, the effective diffusion constant for this equation
can be estimated very simply.

In the absence of surface chemistry phenomena,
we should expect the concentration gradient
to be essentially constant within the pore, and oriented only in the axial direction.
That is, the concentration profile within the pore will be linear,
and at any cross-section there is only a single value of the concentration.
This implies that the problem is very nearly one-dimensional,
as assumed in the definition of $D_{\mathrm{eff}}$.

Defining the concentration gradient within the pore as
\begin{equation}
\vec{\nabla} c = \left(\frac{\mathrm{d}c}{\mathrm{d}s}\right)_{\mathrm{pore}} \hat{n}
\end{equation}

where $\left(\frac{\mathrm{d}c}{\mathrm{d}s}\right)_{\mathrm{pore}}$ is a constant,
the integrated flux within the pore is

\begin{equation}
J_\mathrm{cell} = -D_\mathrm{bulk} \int_\mathrm{cell} \mathrm{d}A\, \left(\frac{\mathrm{d}c}{\mathrm{d}s}\right)_{\mathrm{pore}}
 = -D_\mathrm{bulk} A_\mathrm{pore} \left(\frac{\mathrm{d}c}{\mathrm{d}s}\right)_{\mathrm{pore}}
\end{equation}

Note that the area in this formula is the area of the pore only,
rather than the entire unit cell, because only the pore is within the problem domain
at a cross-section through the pore.

From equation \ref{eq:Deff_Jcell}, we have

\begin{equation}
J_{\mathrm{cell}} = - D_{\mathrm{eff}} A_{\mathrm{cell}} \frac{\Delta c}{\Delta s}
\end{equation}

But with a constant concentration gradient,

\begin{equation}
\frac{\Delta c}{\Delta s} = \left(\frac{\mathrm{d}c}{\mathrm{d}s}\right)_{\mathrm{pore}}
\end{equation}

So,

\begin{equation}
J_{\mathrm{cell}} = - D_{\mathrm{eff}} A_{\mathrm{cell}} \left(\frac{\mathrm{d}c}{\mathrm{d}s}\right)_{\mathrm{pore}}
 = -D_\mathrm{bulk} A_\mathrm{pore} \left(\frac{\mathrm{d}c}{\mathrm{d}s}\right)_{\mathrm{pore}}
\end{equation}

This simplifies to

\begin{equation}
\frac{D_\mathrm{eff}}{D_\mathrm{bulk}} = \frac{A_\mathrm{pore}}{A_\mathrm{cell}}
\end{equation}

which matches the definition of the free volume fraction from Section \ref{subsec:volfrac}, so

\begin{equation}
\frac{D_\mathrm{eff}}{D_\mathrm{bulk}} = \phi
\end{equation}

for unhomogenized Fickian diffusion.
