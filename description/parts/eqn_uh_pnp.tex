%not a stand-alone document; intended for inclusion in a larger document

\section{Unhomogenized Poisson-Nernst-Planck Equation}\label{sec:unhom_pnp}

%-------------------------------------------------------------------------------
\subsection{Governing Differential Equations}\label{subsec:unhom_pnp_gov}


\textcolor{red}{\textbf{TODO}}: need reference for PNP equation

We consider $N_s$ ion species interacting through the an electrostatic potential.
The concentration field of each species is $c_s$, where $s=1 ... N_s$ is the index over the species.
The charge of each ion species is $z_s$, so that the volumetric charge density associated
with ion species $s$ is $z_s c_s$.
The overall volumetric charge density is therefore given by:

\begin{equation}
  \rho = \sum_{s=1}^{N_s}z_s c_s
\end{equation}

From the Poisson equation, the electric potential is therefore governed by:

\begin{equation}\label{eq:Poisson_for_ions}
  \boxed{
    -\epsilon_{0}\epsilon_{r} \nabla^2 \Phi = \sum_{s=1}^{N_s}z_s c_s
  }
\end{equation}

For each ionic species, the change in concentration over time is related to the ion flux as:

\begin{equation}
  \frac{\partial c_s}{\partial t} = - \nabla \cdot \vec{j}_s
\end{equation} 

with ion flux defined as:

\begin{equation}
  \vec{j}_s  = -D_s \left( \nabla c_s + \beta z_s c_s \nabla \Phi \right)
\end{equation}

The governing equation for each ion species is therefore:

\begin{equation}
  \frac{\partial c_s}{\partial t} = \nabla \cdot \left(
  D_s \left( \nabla c_s + \beta z_s c_s \nabla \Phi \right)
  \right)
\end{equation}

Under the assumption that the diffusion constant does not vary spatially,
this becomes:

\begin{equation}
  \frac{\partial c_s}{\partial t} = 
  D_s \left( \nabla^2 c_s + \beta z_s \nabla \cdot \left( c_s \nabla \Phi \right) \right)
\end{equation}

\begin{equation}\label{eq:PNP_timedep}
  \boxed{
    \frac{\partial c_s}{\partial t} = 
    D_s \left( \nabla^2 c_s + \beta z_s \left( \nabla c_s \cdot \nabla \Phi \right)  + \beta z_s c_s \nabla^2 \Phi \right)
  }
\end{equation}

The steady-state condition is therefore governed by

\begin{equation}\label{eq:PNP_steady_state_gov}
  \boxed{
    \nabla^2 c_s + \beta z_s \left( \nabla c_s \cdot \nabla \Phi \right)  + \beta z_s c_s \nabla^2 \Phi = 0
  }
\end{equation}


%-------------------------------------------------------------------------------
\subsection{Weak Form}\label{subsec:unhom_pnp_weak}

The weak form of the Poisson equation is derived by multiplying
Equation \ref{eq:Poisson_for_ions} by a test function $v_\phi$
associated with the electric potential,
and then integrating over the problem domain.

\begin{equation}
  -\epsilon_{0}\epsilon_{r} \int_\Omega \left( \nabla^2 \Phi \right) v_\phi \,\mathrm{d}^3x
   = \sum_{s=1}^{N_s} \int_\Omega z_s c_s v_\phi \,\mathrm{d}^3x
\end{equation}

Performing an integration by parts using the product rule
of Equation \ref{eq:product_rule_divergence}, this becomes

\begin{equation}
  -\epsilon_{0}\epsilon_{r}
  \left( \int_\Omega \nabla \cdot \left( v_\phi \nabla \Phi \right) \,\mathrm{d}^3x
  - \int_\Omega  \left( \nabla \Phi \cdot \nabla v_\phi \right) \,\mathrm{d}^3x \right)
  - \sum_{s=1}^{N_s} \int_\Omega z_s c_s v_\phi \,\mathrm{d}^3x = 0
\end{equation}

Applying the divergence theorem yields

\begin{equation}
  \boxed{
    -\epsilon_{0}\epsilon_{r} \int_{\partial\Omega} \left( \hat{n} \cdot \nabla \Phi \right) v_\phi \,\mathrm{d}s
    + \epsilon_{0}\epsilon_{r} \int_\Omega  \left( \nabla \Phi \cdot \nabla v_\phi \right) \,\mathrm{d}^3x
    - \sum_{s=1}^{N_s} \int_\Omega z_s c_s v_\phi \,\mathrm{d}^3x = 0
  }
\end{equation}

As usual, the boundary term is nonzero only for boundaries with Neumann conditions specified.

The weak form equation for each ion starts from Equation \ref{eq:PNP_timedep},
which is multiplied by a test function $v_s$ associated with the concentration $c_s$,
and then integrated over the problem domain.

\begin{equation}
  \int_\Omega v_s \nabla^2 c_s \,\mathrm{d}^3x 
  + \beta z_s \int_\Omega c_s v_s \nabla^2 \Phi \,\mathrm{d}^3x
  + \beta z_s \int_\Omega  v_s \left( \nabla c_s \cdot \nabla \Phi \right) \,\mathrm{d}^3x
  = \frac{1}{D_s} \int_\Omega \frac{\partial c_s}{\partial t} v_s \,\mathrm{d}^3x
\end{equation}

Performing an integration by parts on the first two terms
using the product rule of Equation \ref{eq:product_rule_divergence}, this becomes

\begin{equation}
  \begin{aligned}
    \int_\Omega \nabla \cdot \left( v_s \nabla c_s \right) \,\mathrm{d}^3x
    - \int_\Omega \left( \nabla c_s \cdot \nabla v_s \right) \,\mathrm{d}^3x \\
    + \beta z_s \left( \int_\Omega \nabla \cdot \left( c_s v_s \nabla \Phi \right) \,\mathrm{d}^3x
    - \int_\Omega \left( \nabla \left( c_s v_s \right)  \cdot \nabla \Phi \right) \,\mathrm{d}^3x \right) 
    + \beta z_s \int_\Omega  v_s \left( \nabla c_s \cdot \nabla \Phi \right) \,\mathrm{d}^3x
    & = & \frac{1}{D_s} \int_\Omega \frac{\partial c_s}{\partial t} v_s \,\mathrm{d}^3x
  \end{aligned}
\end{equation}

Applying the divergence theorem and grouping the factor $\beta z_s$ on the last two terms,
\begin{equation}
  \begin{aligned}
    \int_{\partial\Omega} \left( \hat{n} \cdot \nabla c_s \right) v_s \,\mathrm{d}s
    - \int_\Omega \left( \nabla c_s \cdot \nabla v_s \right) \,\mathrm{d}^3x \\
    + \beta z_s \left( \int_{\partial\Omega} \left( \hat{n} \cdot \nabla \Phi \right) c_s v_s \,\mathrm{d}s
    - \int_\Omega \left( \nabla \left( c_s v_s \right)  \cdot \nabla \Phi \right) \,\mathrm{d}^3x
    + \int_\Omega  v_s \left( \nabla c_s \cdot \nabla \Phi \right) \,\mathrm{d}^3x \right) 
    & = & \frac{1}{D_s} \int_\Omega \frac{\partial c_s}{\partial t} v_s \,\mathrm{d}^3x
  \end{aligned}
\end{equation}

The gradient of a product of two scalars is given by
\begin{equation}
  \nabla(fg) = f(\nabla g) + g (\nabla f)
\end{equation}

Applying this product rule to the necessary terms,

\begin{equation}
  \begin{aligned}
    \int_{\partial\Omega} \left( \hat{n} \cdot \nabla c_s \right) v_s \,\mathrm{d}s
    - \int_\Omega \left( \nabla c_s \cdot \nabla v_s \right) \,\mathrm{d}^3x \\
    + \beta z_s \left( \int_{\partial\Omega} \left( \hat{n} \cdot \nabla \Phi \right) c_s v_s \,\mathrm{d}s 
    - \int_\Omega \left( \left[ c_s \left( \nabla v_s \right) + v_s \left( \nabla c_s \right) \right]  \cdot \nabla \Phi \right) \,\mathrm{d}^3x
    + \int_\Omega  v_s \left( \nabla c_s \cdot \nabla \Phi \right) \,\mathrm{d}^3x \right)
    & =  & \frac{1}{D_s} \int_\Omega \frac{\partial c_s}{\partial t} v_s \,\mathrm{d}^3x
  \end{aligned}
\end{equation}

\begin{equation}
  \begin{aligned}
    \int_{\partial\Omega} \left( \hat{n} \cdot \nabla c_s \right) v_s \,\mathrm{d}s
    - \int_\Omega \left( \nabla c_s \cdot \nabla v_s \right) \,\mathrm{d}^3x \\
    + \beta z_s \left( \int_{\partial\Omega} \left( \hat{n} \cdot \nabla \Phi \right) c_s v_s \,\mathrm{d}s
    - \int_\Omega c_s  \left( \nabla v_s \cdot \nabla \Phi \right) \,\mathrm{d}^3x \right. \\
    \left. - \int_\Omega v_s \left( \nabla c_s \cdot \nabla \Phi \right) \,\mathrm{d}^3x
    + \int_\Omega  v_s \left( \nabla c_s \cdot \nabla \Phi \right) \,\mathrm{d}^3x \right)
    & = & \frac{1}{D_s} \int_\Omega \frac{\partial c_s}{\partial t} v_s \,\mathrm{d}^3x
  \end{aligned}
\end{equation}

\begin{equation}
  \boxed{
    \begin{aligned}
      \int_{\partial\Omega} \left( \hat{n} \cdot \nabla c_s \right) v_s \,\mathrm{d}s
      - \int_\Omega \left( \nabla c_s \cdot \nabla v_s \right) \,\mathrm{d}^3x \\
      + \beta z_s \left( \int_{\partial\Omega} \left( \hat{n} \cdot \nabla \Phi \right) c_s v_s \,\mathrm{d}s
      - \int_\Omega c_s  \left( \nabla v_s \cdot \nabla \Phi \right) \,\mathrm{d}^3x \right)
      & = & \frac{1}{D_s} \int_\Omega \frac{\partial c_s}{\partial t} v_s \,\mathrm{d}^3x
    \end{aligned}
  }
\end{equation}

Note that the boundary term
$\int_{\partial\Omega} \left( \hat{n} \cdot \nabla \Phi \right) c_s v_s \,\mathrm{d}s$
requires some special care.
Because $v_s$ is zero anywhere $c_s$ is known,
the term is zero on the Dirichlet boundaries of $c_s$.
However, the term does not contain the normal derivative of $c_s$,
but rather of $\Phi$.
As such, the term could be nonzero even for Neumann boundaries of $c_s$
that have a prescribed ionic flux of zero.

The other boundary term, $\int_{\partial\Omega} \left( \hat{n} \cdot \nabla c_s \right) v_s \,\mathrm{d}s$
is zero where Dirichlet conditions are specified for $c_s$,
and nonzero only where nonzero Neumann conditions are specified.

The steady-state weak form is

\begin{equation}
  \boxed{
    \begin{aligned}
      \int_{\partial\Omega} \left( \hat{n} \cdot \nabla c_s \right) v_s \,\mathrm{d}s
      - \int_\Omega \left( \nabla c_s \cdot \nabla v_s \right) \,\mathrm{d}^3x 
      + \beta z_s \int_{\partial\Omega} \left( \hat{n} \cdot \nabla \Phi \right) c_s v_s \,\mathrm{d}s
      - \beta z_s \int_\Omega c_s  \left( \nabla v_s \cdot \nabla \Phi \right) \,\mathrm{d}^3x
       = 0
    \end{aligned}
  }
\end{equation}

Note that this equation is nonlinear, as the last two terms include products of the concentration
with the gradient of the electric potential.
Consequently, a nonlinear solution method is required in \texttt{FEniCS}.
In such cases, the definition of a linear and bilinear form, $a$ and $L$,
is not required.
Instead, the entire system is denoted as the equation $FF = 0$.

The complete weak form, $FF$, whose roots is sought, is obtained by adding up the equations for the individual species,
as well as the Poisson equation.

