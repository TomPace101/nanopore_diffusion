%not a stand-alone document; intended for inclusion in a larger document

\section{Time-Domain Simulations}\label{sec:timedomain}

%-------------------------------------------------------------------------------
\subsection{Introduction}\label{subsec:timedomain_intro}

When the conditions are not constant with respect to time,
the time-dependent form of the equation for each ionic species must be solved.
Note that the Poisson equation itself does not contain any time-dependence,
as it is assumed that the electric potential is immediately in equilibrium with the charge locations.
That is, the ions are assumed to move sufficiently slowly that the problem
is electrostatic rather than electrodynamic.

While the diffusion constants $D_s$ fall out of the steady-state equations,
they must be retained for the time-dependent equations.
Thus, we write the steady-state form of the equation for a single ion species as
$WSS_s = 0$,
corresponding to a time-dependent equation

\begin{equation}
  \frac{1}{D_s} \int_\Omega \frac{\partial c_s}{\partial t} v_s \,\mathrm{d}^3x
  = WSS_s
\end{equation}

Prior to adding these equations together, they must each now be rewritten as

\begin{equation}
  \int_\Omega \frac{\partial c_s}{\partial t} v_s \,\mathrm{d}^3x
  = D_s WSS_s
\end{equation}


%-------------------------------------------------------------------------------
\subsection{Backward-Euler Approximation}\label{subsec:backward_euler}

\textcolor{red}{\textbf{TODO}}: rewrite to explain backward vs forwards, and make it clear in the equation.

A simple way to solve the time-dependent equation for each ionic species is to use the Backward-Euler
finite difference approximation for partial derivatives with respect to time.
In this approximation, we already have a solution for $c_s$ at time-step $k$,
which shall be denoted as $c_s^k$.
The solution at time-step $k+1$ shall instead be labeled as simply $c_s$.

The Backward-Euler approximation can then be written as
\begin{equation}
  \frac{\partial c_s}{\partial t} \approx \frac{c_s - c_s^k}{\Delta t}
\end{equation}

Using this approximation, the weak form for each ionic species becomes
\begin{equation}
  \int_\Omega \frac{c_s - c_s^k}{\Delta t} v_s \,\mathrm{d}^3x
  = D_s WSS_s
\end{equation}
 
Rearranging, we have
\begin{equation}
  \boxed{
    \int_\Omega c_s  v_s \,\mathrm{d}^3x
    - \int_\Omega c_s^k  v_s \,\mathrm{d}^3x
    - \Delta t D_s WSS_s = 0
  }
\end{equation}

This form is suitable for the addition of all equations into a single
nonlinear form for solution.
