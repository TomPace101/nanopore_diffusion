%not a stand-alone document; intended for inclusion in a larger document

\section{Result Calculations}

%-------------------------------------------------------------------------------
\subsection{Effective Diffusion Constant}\label{subsec:D_eff}

To compute the effective diffusion constant,
the three-dimensional diffusion problem is considered from the perspective of a simpler one-dimensional diffusion problem.
Specifically, we wish to consider diffusion in the direction across the membrane.
The Fickian diffusion equation (see Section \ref{sec:unhom_fick}) in one dimension can be written as

\begin{equation}\label{eq:fickslaw_1D}
j = - D_{\mathrm{bulk}} \frac{\mathrm{d}c}{\mathrm{d}s}
\end{equation}

where:

$\begin{array}{rcl}
s & = & \text{position within the one-dimensional domain} \\
j & = & j(s) = \text{flux at any point, in units of number of particles per unit area per unit time} \\
D_{\text{bulk}} & = & \text{diffusion constant of the medium, in units of area per unit time} \\
c & = & c(s) = \text{concentration at any point, in units of number of particles per unit volume} \\
\frac{\mathrm{d}c}{\mathrm{d}s} & = & \text{first spatial derivative of concentration}
\end{array}$

The three-dimensional problem is converted to an equivalent one-dimensional problem
by integrating over the area of the unit cell, in the two directions perpendicular to
the one-dimensional diffusion problem.
The total flux across the membrane will be given by:

\begin{equation}
J_{\mathrm{cell}} = - D_{\mathrm{bulk}}\int_{\mathrm{cell}} \mathrm{d}A\, \frac{\partial c}{\partial s}
\end{equation}

We define the effective diffusion constant such that the integrated flux is the same,
when used with an the average concentration gradient across the membrane:

\begin{equation}\label{eq:Deff_Jcell}
J_{\mathrm{cell}} = - D_{\mathrm{eff}} A_{\mathrm{cell}} \frac{\Delta c}{\Delta s}
\end{equation}

where:

$\begin{array}{rcl}
J_{\text{cell}} & = & \text{integral of flux over the pore, in units of number of particles per unit time} \\
A_{\text{cell}} & = & \text{area of unit cell} \\
D_{\text{eff}} & = & \text{unknown effective diffusion constant} \\
\Delta c & = & \text{change in concentration} \\
\Delta s & = & \text{distance over which concentration changes}
\end{array}$

Re-arranging this equation to solve for the unknown $D_{\mathrm{eff}}$, we have:

\begin{equation}\label{eq:Deff_def}
D_{\mathrm{eff}} = - \frac{J_{\mathrm{cell}}}{A_{\mathrm{cell}}} \frac{\Delta s}{\Delta c}
\end{equation}

The integrated flux, $J_{\mathrm{model}}$, is calculated from the model by integrating
the ion flux over a surface parallel to the membrane surface.
The result should be the same for any such surface capturing the full extents of the model.
That is, the integrated flux should be the same when integrating over the model pore as
when integrating over the upgradient or downgradient boundary of the model.

Specifically, the integrated flux is calculated as:

\begin{equation}
J_{\mathrm{model}} = \int_{\mathrm{model}} \mathrm{d}A\, \left(\hat{n} \cdot \vec{j} \right)
 = - D_{\mathrm{bulk}} \int_{\mathrm{model}} \mathrm{d}A\, \left(\hat{n} \cdot \vec{\nabla} c \right)
\end{equation}

where $\hat{n}$ is the directed normal to the surface,
and $c$ is the concentration field found by solving the model.

Because of the use of planes of symmetry (see Section \ref{subsec:geometry}),
the flux obtained by integration is only one quarter of the total for the unit cell.
That is, $J_{\mathrm{cell}} = 4 J_{\mathrm{model}}$.
With $A_{\mathrm{cell}} = 4 Lx Ly$, we have:

\begin{equation}
D_{\mathrm{eff}} = - \frac{4 J_{\mathrm{model}}}{4 Lx Ly} \frac{\Delta s}{\Delta c}
 = - \frac{J_{\mathrm{model}}}{Lx Ly} \frac{\Delta s}{\Delta c}
\end{equation}

For convenience, we define the integral

\begin{equation}
I_{\mathrm{gc}} = \int_{\mathrm{model}} \mathrm{d}A\, \left(\hat{n} \cdot \vec{\nabla} c \right)
\end{equation}

then the flux integral is simply

\begin{equation}
J_{\mathrm{model}} = - D_{\mathrm{bulk}} I_\mathrm{gc}
\end{equation}

and the effective diffusion constant is

\begin{equation}
D_{\mathrm{eff}} = D_{\mathrm{bulk}} \frac{I_\mathrm{gc}}{Lx Ly} \frac{\Delta s}{\Delta c}
\end{equation}

This can also be written as

\begin{equation}
\frac{D_{\mathrm{eff}}}{D_{\mathrm{bulk}}} = \frac{I_\mathrm{gc}}{Lx Ly} \frac{\Delta s}{\Delta c}
\end{equation}

The values of $\Delta c$ and $\Delta s$ are calculated by extracting
the concentration result at two points located symmetrically on opposite sides of the membrane.
The difference in concentration between these two points is $\Delta c$,
and the distance between them is $\Delta s$.

Slightly different results for the value of $D_{\mathrm{eff}}$ could be attained by selecting different
pairs of symmetrically located points.
For consistency, the results here are taken with these two points located
along the centerline of the pore, at both faces of the membrane.

%-------------------------------------------------------------------------------
\subsection{Free Volume Fraction}\label{subsec:volfrac}

The free volume fraction, $\phi$ is defined here as

\begin{equation}
\phi = \frac{\text{pore area}}{\text{unit cell area}}
= \frac{\pi R^2}{Sx Sy} = \frac{\pi R^2}{4 Lx Ly}
\end{equation}
