%not a stand-alone document; intended for inclusion in a larger document

\section{Unhomogenized Smoluchowski Equation}\label{sec:unhom_smol}

%-------------------------------------------------------------------------------
\subsection{Governing Differential Equation}\label{subsec:unhom_smol_gov}

\textcolor{red}{\textbf{TODO}}: need reference for Smoluchowski equation

For an environment free of particle sources and sinks,
the diffusion equation can still be written as:
\begin{equation}
\frac{\partial c}{\partial t} = - \vec{\nabla} \cdot \vec{j}
\end{equation}

but for the Smoluchowski equation, the particle flux is defined as

\begin{equation}
\vec{j} = - D_{\mathrm{bulk}} e^{-\beta \psi} \vec{\nabla} \left( e^{\beta \psi} c \right)
\end{equation}

where:

$\begin{array}{rcl}
c & = & c(x,y,z) = \text{the particle concentration field (number of particles per unit volume)} \\
t & = & \text{time} \\
\vec{j} & = & \vec{j}(x,y,z) = \text{the particle flux field (number of particles per unit area per unit time)} \\
D_{\mathrm{bulk}} & = & \text{the implicit diffusion constant of the solvent (unit area per unit time)} \\
\psi & = & \psi(x,y,z) = \text{the potential field} \\
\beta & = & 1/k_B T
\end{array}$

Combining these equations, the governing equation for steady-state conditions ($\frac{\partial c}{\partial t} = 0$) is

\begin{equation}
\vec{\nabla} \cdot \left( D_{\mathrm{bulk}} e^{-\beta \psi} \vec{\nabla} \left( c e^{\beta \psi} \right) \right) = 0
\end{equation}

\textcolor{red}{\textbf{TODO}}: need reference for Slotboom transformation

The mathematics are simplified considerably by making use of the Slotboom transformation:

\begin{equation}\label{eq:SlotboomTrans}
\begin{array}{rcl}
\overline{D} & = & D_{\mathrm{bulk}} e^{-\beta \psi} \\
\overline{c} & = & c e^{\beta \psi}
\end{array}
\end{equation}

which results in the governing equation

\begin{equation}
\vec{\nabla} \cdot \left( \overline{D} \vec{\nabla} \left( \overline{c} \right) \right) = 0
\end{equation}

Applying the appropriate product rule for the divergence operator in this case,

\begin{equation}\label{eq:SmolSlotboomDiff}
\overline{D} \nabla^2 \overline{c} + \left( \vec{\nabla} \overline{c} \right) \cdot \left( \vec{\nabla} \overline{D} \right) = 0
\end{equation}

For the diffusion of ions in an electrostatic field, the potential field is given by

\begin{equation}
\psi=q \Phi
\end{equation}

where:

$\begin{array}{rcl}
q & = & \text{the electric charge of the ion species} \\
\Phi & = & \text{the electric potential field}
\end{array}$

This means that the Slotboom transformation (Equation \ref{eq:SlotboomTrans}) can be written as

\begin{equation}\label{eq:SlotboomTransWithQ}
\begin{array}{rcl}
\overline{D} & = & D_{\mathrm{bulk}} e^{-\beta q \Phi} \\
\overline{c} & = & c e^{\beta q \Phi}
\end{array}
\end{equation}

Note that the applied boundary conditions must be transformed as well.
This is straightforward for Dirichlet boundary conditions,
as Equation \ref{eq:SlotboomTransWithQ} can be applied directly
to obtain the Dirchlet value of $\overline{c}$ from that of $c$.
Neumann boundary conditions, however, are more complicated.
We wish to specify a value of $\frac{\partial c}{\partial n} = \vec{\nabla} c \cdot \hat{n}$,
but solving the transformed equation requires converting this to specification of the value
$\frac{\partial \overline{c}}{\partial n} = \vec{\nabla} \overline{c} \cdot \hat{n}$.

\begin{equation}
\begin{array}{rcl}
\frac{\partial \overline{c}}{\partial n} & = & \vec{\nabla} \overline{c} \cdot \hat{n} \\
& = & \vec{\nabla} \left( c e^{\beta q \Phi} \right) \cdot \hat{n} \\
& = & \left( c \vec{\nabla} e^{\beta q \Phi} +  e^{\beta q \Phi} \vec{\nabla} c \right) \cdot \hat{n} \\
& = & c \vec{\nabla} \left(e^{\beta q \Phi} \right) \cdot \hat{n}
+ e^{\beta q \Phi} \vec{\nabla} \left( c \right) \cdot \hat{n} \\
& = & c \beta q e^{\beta q \Phi} \vec{\nabla} \left( \Phi \right) \cdot \hat{n}
+ e^{\beta q \Phi} \vec{\nabla} \left( c \right) \cdot \hat{n} \\
& = & c \beta q e^{\beta q \Phi} \frac{\partial \Phi}{\partial n}
+ e^{\beta q \Phi} \frac{\partial c}{\partial n} \\
\frac{\partial \overline{c}}{\partial n} & = & e^{\beta q \Phi}
\left( c \beta q \frac{\partial \Phi}{\partial n} + \frac{\partial c}{\partial n} \right)
\end{array}\end{equation}

Thus, the Neumann boundary conditions on $\overline{c}$ depend not only on
those of $c$, but also on the potential field and its normal derivative at the boundary.
In the case of this model, the normal derivative of $c$ is specified to be zero
at all Neumann boundaries, as is the normal derivative of $\Phi$.
Thus, the boundary condition on $\overline{c}$ is also that its normal derivative is zero
at the Neumann boundaries.

Finally, after solution of the transformed equation,
the ionic concentration field is calculated from the inverse transformation,

\begin{equation}
c = \overline{c} e^{-\beta \psi} = \overline{c} e^{-\beta q \Phi}
\end{equation}

%-------------------------------------------------------------------------------
\subsection{Weak Form}\label{subsec:unhom_smol_weak}

Starting from the governing differential equation (Equation \ref{eq:SmolSlotboomDiff}),
multiplying by a test function $v$ and integrating over the problem domain, we have

\begin{equation}\label{eq:SmolWeakStart}
\int_{\Omega} \overline{D} v \left( \nabla^2 \overline{c} \right) \,\mathrm{d}^3x 
+ \int_{\Omega} v \left( \vec{\nabla} \overline{c} \right) \cdot \left( \vec{\nabla} \overline{D} \right) \,\mathrm{d}^3x = 0
\end{equation}

Using the same product rule as Equation \ref{eq:product_rule_divergence}, 
we have

\begin{equation}
\vec{\nabla} \cdot \left( \overline{D} v \vec{\nabla} \overline{c} \right) =
\overline{D} v \left(\vec{\nabla} \cdot \vec{\nabla} \overline{c} \right)
+ \vec{\nabla}\overline{c} \cdot \vec{\nabla} \left( \overline{D} v \right) =
\overline{D} v \left(\nabla^2 \overline{c} \right)
+ \vec{\nabla}\overline{c} \cdot \vec{\nabla} \left( \overline{D} v \right)
\end{equation}

And so, applying the product rule for the gradient of two scalars,
\begin{equation}
\overline{D} v \left(\nabla^2 \overline{c} \right) =
\vec{\nabla} \cdot \left( \overline{D} v \vec{\nabla} \overline{c} \right)
- \vec{\nabla}\overline{c} \cdot \left( \overline{D} \vec{\nabla} v + v \vec{\nabla} \overline{D} \right)
\end{equation}

\begin{equation}
\overline{D} v \left(\nabla^2 \overline{c} \right) =
\vec{\nabla} \cdot \left( \overline{D} v \vec{\nabla} \overline{c} \right)
- \overline{D} \left(\vec{\nabla}\overline{c}\right) \cdot \left(\vec{\nabla}v\right)
- v \left(\vec{\nabla}\overline{c}\right) \cdot \left(\vec{\nabla}\overline{D}\right)
\end{equation}

Taking the integral over the problem domain, this gives us

\begin{equation}
\int_{\Omega} \overline{D} v \left(\nabla^2 \overline{c} \right) \,\mathrm{d}^3x=
\int_{\Omega} \vec{\nabla} \cdot \left( \overline{D} v \vec{\nabla} \overline{c} \right) \,\mathrm{d}^3x
- \int_{\Omega} \overline{D} \left(\vec{\nabla}\overline{c}\right) \cdot \left(\vec{\nabla}v\right)\,\mathrm{d}^3x
- \int_{\Omega} v \left(\vec{\nabla}\overline{c}\right) \cdot \left(\vec{\nabla}\overline{D}\right) \,\mathrm{d}^3x
\end{equation}

Substituting this into Equation \ref{eq:SmolWeakStart},
the governing equation becomes

\begin{equation}
\int_{\Omega} \vec{\nabla} \cdot \left( \overline{D} v \vec{\nabla} \overline{c} \right) \,\mathrm{d}^3x
- \int_{\Omega} \overline{D} \left(\vec{\nabla}\overline{c}\right) \cdot \left(\vec{\nabla}v\right)\,\mathrm{d}^3x
- \int_{\Omega} v \left(\vec{\nabla}\overline{c}\right) \cdot \left(\vec{\nabla}\overline{D}\right) \,\mathrm{d}^3x
+ \int_{\Omega} v \left( \vec{\nabla} \overline{c} \right) \cdot \left( \vec{\nabla} \overline{D} \right) \,\mathrm{d}^3x = 0
\end{equation}

The final two terms cancel one another.
Using the divergence theorem (Gauss's theorem), the result is
\begin{equation}
\int_{\partial\Omega} \overline{D} v \left(\hat{n} \cdot \vec{\nabla}\overline{c}\right) \,\mathrm{d}s
- \int_{\Omega} \overline{D} \left(\vec{\nabla}\overline{c}\right) \cdot \left(\vec{\nabla}v\right)\,\mathrm{d}^3x
= 0
\end{equation}

The boundary term separates into the Dirichlet boundaries (where $\overline{c}$ is known and therefore $v=0$),
and the Neumann boundaries, as discussed in Section \ref{subsec:unhom_smol_gov},
where it was noted that $\hat{n} \cdot \vec{\nabla} \overline{c} = 0$ at all points on the von Neumann boundaries.
Thus, the boundary term is zero in its entirety,
and so the weak form of the transformed Smoluchowski equation in this case is

\begin{equation}
\int_{\Omega} \overline{D} \left(\vec{\nabla}\overline{c}\right) \cdot \left(\vec{\nabla}v\right)\,\mathrm{d}^3x = 0
\end{equation}

In terms of \texttt{FEniCS}, this means that the bilinear form is

\begin{equation}
a(\overline{c},v)= \overline{D} \left(\vec{\nabla}\overline{c}\right) \cdot \left(\vec{\nabla}v\right)\,\mathrm{d}^3x
\end{equation}

and the linear form is constant, zero.


%-------------------------------------------------------------------------------
\subsection{Applied Potential}\label{subsec:unhom_smol_potential}

To solve the Smoluchowski diffusion equation, the electric potential $\Phi$ must be given.
Here, we will calculate the electric potential using the linearized Poisson-Boltzmann equation
(Equation 15-29 in \cite{McQuarrie-StatMech}).

\begin{equation}\label{eq:PoissonBoltzmann}
\nabla^2 \Phi = \kappa^2 \Phi
\end{equation}

where the constant $\kappa$ has dimensions of inverse length.
It is also noted in \cite{McQuarrie-StatMech} that this equation becomes exact
in the limit $\kappa \rightarrow 0$,
which corresponds to very low ionic concentrations.

The length $\lambda_D = 1/\kappa$ is commonly known as the Debye length.
A formula for the calculation of $\kappa^2$ is provided in \cite{McQuarrie-StatMech},
but it requires knowing the spatial distribution of the ions.
Instead, we will use the Debye length as a property of the problem domain.
Then, with appropriate boundary conditions,
the electric potential can be found by Finite Element solution of Equation \ref{eq:PoissonBoltzmann}.

\textcolor{red}{\textbf{TODO}}: show sketch of electrostatic boundary conditions

To solve for the potential in \texttt{FEniCS},
a weak form of the Poisson-Boltzmann equation (Equation \ref{eq:PoissonBoltzmann}) is required.
Multiplying by the test function $v$ and integrating over the problem domain, we obtain

\begin{equation}
\int_{\Omega} \left(\nabla^2 \Phi \right) v \,\mathrm{d}^3x = \frac{1}{\lambda_D^2} \int_{\Omega} \Phi v \,\mathrm{d}^3x
\end{equation}

Using the same product rule as Equation \ref{eq:product_rule_divergence} for integration by parts, this becomes
\begin{equation}
\int_{\Omega} \vec{\nabla} \cdot \left( v \vec{\nabla} \Phi \right) \,\mathrm{d}^3x
- \int_{\Omega} \vec{\nabla}\Phi \cdot \vec{\nabla}v \,\mathrm{d}^3x
= \frac{1}{\lambda_D^2} \int_{\Omega} \Phi v \,\mathrm{d}^3x
\end{equation}

Applying the divergence theorem (Gauss's theorem),
\begin{equation}
\int_{\partial\Omega} \left( \hat{n} \cdot \vec{\nabla} \Phi \right) v \,\mathrm{d}s
- \int_{\Omega} \vec{\nabla}\Phi \cdot \vec{\nabla}v \,\mathrm{d}^3x
= \frac{1}{\lambda_D^2} \int_{\Omega} \Phi v \,\mathrm{d}^3x
\end{equation}

The boundary term separates into the Dirichlet boundaries (where $\Phi$ is known and therefore $v=0$),
and the Neumann boundaries, which in this case will have zero electric field normal to the surface.
This condition requires that the derivative of the potential normal to the surface is zero,
and so $\hat{n} \cdot \vec{\nabla} \Phi = 0$ at all points on the von Neumann boundaries.
Thus, the boundary term is zero in its entirety, and the equation becomes

\begin{equation}
\frac{1}{\lambda_D^2} \int_{\Omega} \Phi v \,\mathrm{d}^3x
+ \int_{\Omega} \vec{\nabla}\Phi \cdot \vec{\nabla}v \,\mathrm{d}^3x
= 0
\end{equation}

In terms of \texttt{FEniCS}, this means that the bilinear form is

\begin{equation}
a(\Phi,v)=\left(\frac{1}{\lambda_D^2} \Phi v  + \vec{\nabla}\Phi \cdot \vec{\nabla}v \right) \,\mathrm{d}^3x
\end{equation}

and the linear form is constant, zero.

%-------------------------------------------------------------------------------
\subsection{Expected Results}\label{subsec:unhom_smol_expected}

The Smoluchowski diffusion equation does not model any interaction between the ions themselves.
Therefore, with an applied potential of zero,
the solution should be the same as for the
unhomogenized Fickian diffusion equation (Section \ref{subsec:unhom_fick_expected}).

%-------------------------------------------------------------------------------
\subsection{Results from Unhomogenized Smoluchowski Equation}\label{subsec:res_unhom_smol}

\textcolor{red}{\textbf{TODO}}

\begin{figure}[H]
\centering
\maybeincludegraphics[width=1\textwidth]{../data/postproc/thin-shot/ratio_vs_phi.pdf}
\caption{Result of Smoluchowski Diffusion Analysis}
\label{fig:uh_smol_D_vs_phi}
\end{figure}
