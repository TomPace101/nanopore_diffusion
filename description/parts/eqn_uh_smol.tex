%not a stand-alone document; intended for inclusion in a larger document

\section{Unhomogenized Smoluchowski Equation}\label{sec:unhom_smol}

%-------------------------------------------------------------------------------
\subsection{Governing Differential Equation}\label{subsec:unhom_smol_gov}

\textcolor{red}{\textbf{TODO}}

For diffusion of ions, the potential field is given by

\begin{equation}
\psi=q_i \Phi
\end{equation}

where:

$\begin{array}{rcl}
\psi & = & \psi(x,y,z) = \text{the potential field} \\
q_i & = & \text{the electric charge of ion species} i\\
\Phi & = & \text{the electric potential field}
\end{array}$

\textcolor{red}{\textbf{TODO}}

%-------------------------------------------------------------------------------
\subsection{Weak Form}\label{subsec:unhom_smol_weak}

\textcolor{red}{\textbf{TODO}}

%-------------------------------------------------------------------------------
\subsection{Applied Potential}\label{subsec:unhom_smol_potential}

To solve the Smoluchowski diffusion equation, the electric potential $\Phi$ must be given.
Here, we will calculate the electric potential using the linearized Poisson-Boltzmann equation
(Equation 15-29 in \cite{McQuarrie-StatMech}).

\begin{equation}\label{eq:PoissonBoltzmann}
\nabla^2 \Phi = \kappa^2 \Phi
\end{equation}

where the constant $\kappa$ has dimensions of inverse length.
It is also noted in \cite{McQuarrie-StatMech} that this equation becomes exact
in the limit $\kappa \rightarrow 0$,
which corresponds to very low ionic concentrations.

The length $\lambda_D = 1/\kappa$ is commonly known as the Debye length.
A formula for the calculation of $\kappa^2$ is provided in \cite{McQuarrie-StatMech},
but it requires knowing the spatial distribution of the ions.
Instead, we will use the Debye length as a property of the problem domain.
Then, with appropriate boundary conditions,
the electric potential can be found by Finite Element solution of Equation \ref{eq:PoissonBoltzmann}.

\textcolor{red}{\textbf{TODO}}: show sketch of electrostatic boundary conditions

To solve for the potential in \texttt{FEniCS},
a weak form of the Poisson-Boltzmann equation (Equation \ref{eq:PoissonBoltzmann}) is required.
Multiplying by the test function $v$ and integrating over the problem domain, we obtain

\begin{equation}
\int_{\Omega} \left(\nabla^2 \Phi \right) v \,\mathrm{d}^3x = \frac{1}{\lambda_D^2} \int_{\Omega} \Phi v \,\mathrm{d}^3x
\end{equation}

Using the same product rule as Equation \ref{eq:product_rule_divergence}, this becomes
\begin{equation}
\int_{\Omega} \vec{\nabla} \cdot \left( v \vec{\nabla} \Phi \right) \,\mathrm{d}^3x
- \int_{\Omega} \vec{\nabla}\Phi \cdot \vec{\nabla}v \,\mathrm{d}^3x
= \frac{1}{\lambda_D^2} \int_{\Omega} \Phi v \,\mathrm{d}^3x
\end{equation}

Applying the divergence theorem (Gauss's theorem),
\begin{equation}
\int_{\partial\Omega} \left( \hat{n} \cdot \vec{\nabla} \Phi \right) v \,\mathrm{d}s
- \int_{\Omega} \vec{\nabla}\Phi \cdot \vec{\nabla}v \,\mathrm{d}^3x
= \frac{1}{\lambda_D^2} \int_{\Omega} \Phi v \,\mathrm{d}^3x
\end{equation}

The boundary term separates into the Dirichlet boundaries (where $\Phi$ is known and therefore $v=0$),
and the Neumann boundaries, which in this case will have zero electric field normal to the surface.
This condition requires that the derivative of the potential normal to the surface is zero,
and so $\hat{n} \cdot \vec{\nabla} c = 0$ at all points on the von Neumann boundaries.
Thus, the boundary term is zero in its entirety, and the equation becomes

\begin{equation}
\frac{1}{\lambda_D^2} \int_{\Omega} \Phi v \,\mathrm{d}^3x
+ \int_{\Omega} \vec{\nabla}\Phi \cdot \vec{\nabla}v \,\mathrm{d}^3x
= 0
\end{equation}

In terms of \texttt{FEniCS}, this means that the bilinear form is

\begin{equation}
a(\Phi,v)=\left(\frac{1}{\lambda_D^2} \Phi v  + \vec{\nabla}\Phi \cdot \vec{\nabla}v \right) \,\mathrm{d}^3x
\end{equation}

and the linear form is constant, zero.

%-------------------------------------------------------------------------------
\subsection{Expected Results}\label{subsec:unhom_smol_expected}

The Smoluchowski diffusion equation does not model any interaction between the ions themselves.
Therefore, with an applied potential of zero,
the solution should be the same as for the
unhomogenized Fickian diffusion equation (Section \ref{subsec:unhom_fick_expected}).

